% Include the average room temperature and humidity during lay-up.  Would the quality of the manufactured specimens be affected by the room temperature and humidity levels?  Explain.

The lab group did not record the average room temperature and humidity readings during the lay-up process. Thus, we presume that the the lay-up process was completed under standard lab conditions: 23$\degree$C and 40\% relative humidity. Room temperature and humidity levels are influential on the quality of manufactured specimens. During storage, the composite prepreg is kept at freezing temperatures. As the temperature increases, the epoxy will begin to crosslink. While this behavior is desired during the curing process, it is important to to avoid curing the composites prematurely from a high room temperature. Note that premature crosslinking reduces the quality of the composite specimen. It is also important to regulate the humidity level during the layup process to prevent the specimen from absorbing additional moisture. Moisture absorption can lead to the formation of more voids, which are unfilled pores in the specimen. These voids are major manufacturing defects that reduce the performance of the specimen in terms of mechanical properties and lifespan.