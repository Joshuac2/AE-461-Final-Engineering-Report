% What would happen if the cure temperature was raised to a higher temperature?  How about a lower temperature?
\tab The DA 409U/G35 150 prepreg used in this experiment requires temperature cure to properly complete the manufacturing process and create the epoxy cross-links.  The temperature curing process, a process henceforth defined as thermosetting, is a very critical stage wherein temperature and pressure is applied to the specimen through a hot press to begin the polymerization, or cross-linking of the resin matrix.  In this experiment, a prepreg was used to form the specimen, which is useful for generating small, thin-skinned specimen for testing or thin walled applications, but is prone to premature curing depending on the local room temperature.

During the assembling process (\textit{lay-up stage}) of the composite specimen, an observation was made such that the curing process was able to begin prematurely due to the local heat generated by physical touch.  The lab manual warned of consequences directly linked to premature curing such as reduced mechanical quality due to malformed epoxy cross-links.  As such, rapid freezing or chilling techniques using condensed air or placement near a cold thermal-sink were applied to mitigate premature cure.  When the specimen were fully assembled, they were placed into molds (\textit{pre-cure stage}) with peeler ply, release film, and bleeder cloths.  When completed, the composites were ready for the thermosetting process.

\newpage
It is important to note that the temperature setting for the curing stage was controlled by the teaching assistants; and, variation of the curing temperature was not a focus of this experiment.  The temperature and pressure applied for the curing process was selected and performed according to the Carver programmable hot press with a cure cycle programmed by the lab staff.  If the temperature were to increase, the development of the cross-links would be much faster and rapid, which may improve the cure by preventing resin bleed and higher shrinkage would occur.  Higher shrinkage, depending on the application, may be desired due to the improvement in the material's thermal and electrical conductivity \cite{curematters}.  Higher temperature is not always good, though, as excessive temperature can result in a severe degradation of resin matrix.\cite{boeingcuring}  A lower temperature setting for the cure process of the specimen might result in the desired design performance due to a lower setting of the cross-link structures.  Variation of curing temperature may result in a variation in the quality of the cross-link structures, resin leakage, or material shrinkage.