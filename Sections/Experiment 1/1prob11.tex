% Explain why the heating rate is much more linear than the cooling rate.

The curing cycle involves a rise in temperature for polymerization and then a cooling period afterwards. The temperatures that the specimen experiences during the cycle are determined by the hot press platens. It is important to note that the platens do not heat and cool at the same rate. The heating rate is much more linear than the cooling rate, which can be attributed to the processes by which the platens are heated and cooled. The heating rate is controlled through a process known as Joule heating. In this process, an electrical current is passed through a conductor, which produces heat. The heating rate can then be monitored using a thermocouple and actively controlled through the electric current. In contrast, the platens are cooled using flowing chilled water. In this case, not only is the cooling rate not actively controlled, but the flow rate is also constant. The cooling rate of the platens slows down over time since the temperature of the platens decreases, while the temperature of the chilled flowing water is constant. This results in a nonlinear cooling rate.