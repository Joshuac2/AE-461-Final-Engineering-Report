% What would happen if the applied pressure was raised to a higher pressure? How about a lower pressure?
An important aspect of the curing process is the pressure applied.  Different thermoset composite materials require different pairings of temperature and pressure applied during the curing process.  Earlier, it was discussed how an increase or decrease in temperature can directly impact the quality of the cured specimen.  Similarly, the applied pressure has a major role in determining the material quality.  An increase in pressure may result in a rapid expansion of resin, which has the consequence of malformed the cross-links.  Similarly, a lower pressure may result in a longer required cure time as the thermosetting procedure would be decelerated by the lower pressure.  It is important to properly match the pressure and temperature cycle requirements per the different types of composites in order to successfully obtain a composite with the desired mechanical properties.