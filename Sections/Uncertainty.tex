Sources of uncertainty in this experiment consist of the measurement devices used and calculated quantities presented. In the case of measured values, the uncertainty is an estimate of the instrument using the experiment data given. For calculations, the method of uncertainty propagation is used to find the uncertainty of each variable. Zero uncertainty is assumed for material properties given in the lab manual. Since uncertainty in this experiment is due to the devices used to gather data, increasing the accuracy of the measurement instruments will reduce the uncertainty. User error can be lowered as well by taking repeated measurements, such as when finding specimen dimensions or mass. Sample calculations of uncertainty propagation are documented in the Appendix C.       
\begin{table}[!h]
    \centering
    \caption{Uncertainty \cite{labmanual}}
    \begin{tabular}{|c|c|c|}\toprule
        \multicolumn{3}{c}{\textbf{Uncertainty Estimation}} \\ \midrule
        \textbf{Data} & \textbf{Uncertainty} & \textbf{Reasoning} \\ \hline\hline
        Load                        & 0.00005 N  & Estimate of measurement device \\\hline
        Specimen Length         & 0.00005 m  & Estimate of measurement device  \\\hline
        Specimen Width         & 0.00000005 m  & Estimate of measurement device  \\\hline
        Specimen Thickness         & 0.00000005 m  & Estimate of measurement device  \\\hline
        Specimen Mass             &  0.00005 kg & Estimate of measurement device\\\hline
        Strain Extensometer         & 0.00005 & Estimate of measurement device  \\\hline
        Strain DIC                  & 0.00005 & Estimate of measurement device  \\\hline
        Specimen Density            & 8.065 e-9 kg/m$^3$ & Calculated using uncertainty propagation  \\\hline
        Specimen Volume             & 0.0018 m$^3$ & Calculated using uncertainty propagation  \\\hline
        Specimen Volume Shrinkage   & 0.0255 m$^3$ & Calculated using uncertainty propagation  \\\hline
        Elastic Modulus             &  & Calculated using uncertainty propagation  \\\hline
        $E_{x}$                     &  & Calculated using uncertainty propagation  \\\hline
        $\nu_{xy}$                  &  & Calculated using uncertainty propagation  \\\hline
        $E_{y}$                     &  & Calculated using uncertainty propagation  \\\hline
        $G_{xy}$                    &  & Calculated using uncertainty propagation  \\\hline
        %$c$                         &  & Calculated using uncertainty propagation  \\\hline
        %$\overline{Q}_{11}$         &  & Calculated using uncertainty propagation\\\hline      
        %$\overline{Q}_{22}$         &  & Calculated using uncertainty propagation\\\hline                        
        %$\overline{Q}_{12}$         &  & Calculated using uncertainty propagation\\\hline                        
        %$\overline{Q}_{66}$         &  & Calculated using uncertainty propagation\\\hline               
        %${Q}_{16}$                  &  & Calculated using uncertainty propagation\\\hline                        
        %${Q}_{26}$                  &  & Calculated using uncertainty propagation\\\hline                        
        Theoretical Failure Load    & 0 Pa & Calculated using uncertainty propagation\\\hline                        
        %    & & \\\hline                        
         %   & & \\\hline                        
          %  & & \\\hline                   
        Lab Relative Humidity       & 0.1$\%$ & Estimate of measurement device    \\\hline
        Lab Temperature             & 0.1$\degree$ & Estimate of measurement device   \\\bottomrule
    \end{tabular}
    \label{tab:uncertainty}
\end{table}