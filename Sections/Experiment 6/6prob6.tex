% Compute the theoretical failure loads of the 0 deg and 90 deg composite specimens using equation (6.54). Compute the failure load for the off-axis +/-45 deg composite specimen using the same equation (Hint: Assume theta = +45 deg only).  Compare these results with those obtained experimentally.
\begin{table}[!h]
    \centering
    \caption{Theoretical and Experimental Tensile Failure Loads \cite{labmanual}}
    \begin{tabular}{|C{1in}|c|c|}\toprule
        \multicolumn{3}{c}{\textbf{Tensile Failure Loads}} \\ \midrule
        \textbf{Laminate\newline Orientation} & \textbf{Theoretical} & \textbf{Experimental} \\ \hline\hline
        $0\degree$  & 1.930 GPa  & 1.156 GPa \\\hline
        $45\degree$ & 0.0668 GPa & 0.1386 GPa \\\hline
        $90\degree$ & 0.040 GPa  & 0.0059 GPa \\\bottomrule
    \end{tabular}
    \label{tab:failureloads}
\end{table}

The experimental failure loads are lower for the $0\degree$ and $90\degree$ laminate orientations. However, the experimental failure load is higher for the $45\degree$ specimen. 

