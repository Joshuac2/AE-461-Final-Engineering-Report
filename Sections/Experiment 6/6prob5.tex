% Compare the experimental modulus (E1) for the composite specimens (as measured with the digital extensometer and from your DIC calculations) and the predictions determined in (4).

\begin{table}[!h]
    \centering
    \caption{Comparison of E$_{1}$ \cite{labmanual}}
    \begin{tabular}{|C{1in}|c|c|c|}\toprule
        \multicolumn{4}{c}{\textbf{Theoretical and Experimental Modulus}} \\ \midrule
        \textbf{Laminate\newline Orientation} & \textbf{Theoretical} & \textbf{DIC} & \textbf{Extensometer} \\ \hline\hline
         $0\degree$ & 130.37 GPa & 119.84 GPa & 166 GPa \\\hline
        $45\degree$ & 5.918 GPa & 10.9 GPa & 18 GPa  \\\hline
        $90\degree$ & 5.484 GPa & 2.69 GPa & 2.59 GPa \\\bottomrule
    \end{tabular}
    \label{tab:e1}
\end{table}

Compared to the theoretical values, the experimental modulus from DIC is lower for the $0\degree$ and $90\degree$ laminate orientations but greater for the $45\degree$ orientation. The experimental modulus from the extensometer is larger than the other results for both the $0\degree$ and $45\degree$ specimens but is the lowest for the $90\degree$ orientation.