% Compute the specific elastic modulus and specific strength of all the specimens using the composite density data generated for each cured specimen in the manufacturing lab, Lab #1.  Compare the values of the three specimens to each another.
\begin{table}[!h]
    \centering
    \caption{Specimen specific properties \cite{labmanual}}
    \begin{tabular}{|C{1in}|c|c|}\toprule
        \multicolumn{3}{c}{\textbf{Specific Properties}} \\ \midrule
        \textbf{Laminate\newline Orientation} & \textbf{Specific Elastic Modulus} & \textbf{Specific Strength} \\ \hline\hline
         $0\degree$ & 76.21 MPa  & 735.1 $\frac{kPa}{kg/m^{3}}$ \\\hline
        $45\degree$ & 6.957 MPa & 88.49 $\frac{kPa}{kg/m^{3}}$  \\\hline
        $90\degree$ & 1.714 MPa & 3.749 $\frac{kPa}{kg/m^{3}}$  \\\bottomrule
    \end{tabular}
    \label{tab:specprop}
\end{table}

The specific elastic modulus and specific strength of the $0\degree$ laminate orientation specimen is much greater than the other two orientations. Even so, the $45\degree$ specimen still has significantly greater specific elastic modulus and specific strength than the $90\degree$ orientation.