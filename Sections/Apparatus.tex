\subsection{Manufacturing Apparatus}
\tab A set DA 409U/G35 150 prepeg sheets were provided for the purposes of designing and forming both the $0\degree$ and $45\degree$ specimens.  A sharp utility knife was provided to slice the sheets into the proper dimensions.  Due to the high sensitivity to local humidity and temperature, the manufacturing process was done in a large, well ventilated room with an analog hygrometer/digital thermometer.  The analog hygrometer had a relative humidity accuracy of 2.5\% and resolution of 1.0\%; the digital thermometer had an accuracy of $1.5\degree$F and resolution of $0.1\degree$F.  For the purpose of weighing the specimen prior to cure, an electronic top loading balance scale was provided with a linearity of $\pm 0.001$ grams and $4.0$ second stabilization time.  During the measuring, cutting, and assembling of the specimen, slightly unconventional methods of chilling the composite were used to rapidly decrease the local temperature of the material to prevent premature curing and mitigate the slow increase of void content in the composite matrix.  Some methods included compressed air cans or thermal cold-sinks.

\begin{table}[h!]
    \centering
    \caption{Equipment and Specifications for Specimen Manufacturing}
    \begin{tabular}{|c|C{0.25\textwidth}|C{0.25\textwidth}|C{0.12\textwidth}|C{0.2\textwidth}|}\toprule
        \textbf{No.} & \textbf{Item} & \textbf{Description} & \textbf{Accuracy} & \textbf{Use in Experiment} \\ \midrule
        \textbf{1} & DA 409U/G35\newline 150 prepreg & Unidirectional reinforcement,\newline G35 graphite fibers,\newline 409 epoxy matrix, Prepreg planar density of $\rho = 150 g/m^2$ & N/A & Slices were cut out of preform sheets to construct polymer composite specimen \\ \hline
        \textbf{2} & Traceable\textsuperscript{\tiny\textregistered} Analog Hygrometer/ \newline Digital Thermometer & 5.0 inch dial,\newline RH range: 0\% - 100\%.\newline Thermometer range: $23.0\degree-131.0\degree$F & RH $2.5\%$, T $1.5\degree$F & Placed in layup, weighing, and hot press rooms. \\\hline
        \textbf{3} & Electronic Top Loading Balance & Highly sensitive scale & $\pm0.001$ g & Weigh composite specimen \& bleeder cloth before and after cure \\\hline
    \end{tabular}
    \label{tab:equipmentpart1}
\end{table}

\begin{table}[h!]
    \centering
    \begin{tabular}{|c|C{0.25\textwidth}|C{0.25\textwidth}|C{0.12\textwidth}|C{0.2\textwidth}|}
        \multicolumn{5}{c}{\textbf{Table 1} (\textit{continued})\textbf{:} \textbf{Lay-up Materials}} \\\midrule
        \textbf{4} & AirTech N10 Bleeder Cloth & Absorb excess resin \& \newline Provide vacuum path to remove volatiles & N/A & Bleeder cloth to absorb resin \\\hline
        \textbf{5} & Release-Ease 234 TFP Peel Ply & Teflon coated to prevent resin sticking & N/A & Porosity allowed uniform displacement of resin while preventing undesired adhesion \\\hline
        \textbf{6} & A4000R Non-perforated TFP Film & Placed between press and layup materials & N/A & Prevents lay-up materials from adhering to the mold and tool \\\hline
        \textbf{7} & 2024-T3 Aluminium Mold & Three $1.0 x 7.0$ slots for lay-up material & N/A & Holds the lay-up material while placed in hot press \\\hline
        \textbf{8} & Carver Programmable Hot press & Electrical Platens,\newline Max Temperature $850\degree$F,\newline Max Pressure 24 tons & N/A & Hot press to thermoset specimen \\\hline
    \end{tabular}
    \label{tab:equipmentpart2}
\end{table}

\hl{ADD PICTURE CRAP STUFF}

\subsection{Mechanical Property Testing Apparatus}
\hl{ADD TEXT JOSH}

\begin{table}[h!]
    \centering
    \caption{Equipment and Specifications for Uniaxial Tension Testing}
    \begin{tabular}{|c|C{0.25\textwidth}|C{0.25\textwidth}|C{0.12\textwidth}|C{0.2\textwidth}|}\toprule
        \textbf{No.} & \textbf{Item} & \textbf{Description} & \textbf{Accuracy} & \textbf{Use in Experiment} \\ \midrule
        \textbf{1} & Instron model 4483 & Screw-driven tension compression tests\newline Max load 20 kips & N/A & Used to exert tension load on specimen \\\hline
        \textbf{2} & Instron model 2630-100 static Extensometer & 1.0 inch gauge length & ECA:\newline $\pm0.06$ FRO \cite{extensometer} & Measure specimen strain \\\hline
    \end{tabular}
    \label{tab:equipmentpart3}
\end{table}

\hl{ADD PICTURES}